\newcommand{\cvut}{České vysoké učení technické v~Praze}
\newcommand{\fjfi}{Fakulta jaderná a fyzikálně inženýrská}
\newcommand{\km}{Katedra matematiky}
\newcommand{\obor}{Inženýrská Informatika}
\newcommand{\zamereni}{Softwarové inženýrství a matematická informatika}

\newcommand{\nazevcz}{Prostředí pro implementaci algoritmů Bayesovské filtrace}
\newcommand{\nazeven}{Implementation environment for Bayesian filtering algorithms}
\newcommand{\autor}{Matěj Laitl}
\newcommand{\rok}{2011}
\newcommand{\vedouci}{Ing. Václav Šmídl, Ph.D.}

\newcommand{\pracovisteVed}{Oddělení adaptivních systémů \\
	Ústav teorie informace a automatizace \\
	Akademie věd České republiky}
\newcommand{\konzultant}{---}
\newcommand{\pracovisteKonz}{}

\newcommand{\klicova}{Bayesovská filtrace, softwarová analýza, Python, Cython} % TODO: dalsi?
\newcommand{\keyword}{Bayesian filtering, software analysis, Python, Cython}
% Abstrakt práce: (cca 7 vět, min. 80 slov)
\newcommand{\abstrCZ}{Bayesovská filtrace je velmi použitelný přístup k odhadování dynamických
systémů, který má mnoho aplikací v robotice, environmentálních simulacích a dalších oborech. Tato
práce předkládá popis Kalmanova filtru, částicového filtru a marginalizovaného částicového filtru,
který využívá myšlenky obou předchozích. Poté je provedena softwarová analýza, která má za úkol
určit nejvhodnější implementační prostředí pro softwarovou knihovnu poskytující metody Bayesovské
filtrace. Nejprve jsou posouzeny obecné přístupy k vývoji software, z nichž je jako nejpoužitelnější
zvolen objektově orientovaný, následuje porovnání programovacích jazyků C++, MATLAB a Python.
Python zvítězí a v je něm napsána knihovna pro Bayesovskou filtraci nazvaná PyBayes; nakonec je
dosaženo výrazného zrychlení Pythonového kódu použitím Cythonu.}
\newcommand{\abstrEN}{Bayesian filtering is a vital approach to dynamic system estimation that has
wide applications in robotics, environmental simulations and much more. This text briefly introduces
the Kalman filter, the particle filter and the marginalized particle filter which combines ideas of
both. Software analysis is performed with the aim to identify the most suitable implementation
environment for a software library for Bayesian filtering. Various paradigms of software development
and are discussed among which the object-oriented programming is chosen; C++, MATLAB and Python
languages are evaluated. Python is determined most appropriate and a Python software library named
PyBayes is developed; dramatic performance gains are measured when Cython is used to speed up
Python.}

%%% zde zacina kresleni dokumentu

% titulní strana
\thispagestyle{empty}

\begin{center}
	{\Large  \bf  \cvut\\[2mm] \fjfi }
	\vspace{10mm}

	\begin{tabular}{c}
	{\bf \km}\\
	{\bf Obor: \obor}\\
	{\bf Zaměření: \zamereni}
	\end{tabular}

	\vspace{10mm} \epsfysize=20mm  \epsffile{cvut-logo-bw-600} \vspace{15mm}

	{\LARGE
	\textbf{\nazevcz}
	\par}

	\vspace{5mm}

	{\LARGE
	\textbf{\nazeven}
	\par}

	\vspace{30mm}
	{\Large BAKALÁŘSKÁ PRÁCE}

\end{center}

\vfill
{\large
\begin{tabular}{rl}
Vypracoval: & \autor\\
Vedoucí práce: & \vedouci\\
Rok: & \rok
\end{tabular}
}

% zadání bakalářské práce
\newpage
\thispagestyle{empty} Před svázáním místo téhle stránky \fbox{vložíte zadání práce} s podpisem
děkana (bude to jediný oboustranný list ve Vaší práci) !!!!

% prohlášení
\newpage
\thispagestyle{empty}
~
\vfill


{\noindent}{\LARGE Prohlášení}

\vspace{0.5cm}
Prohlašuji, že jsem předloženou práci vypracoval samostatně a že jsem uvedl veškerou použitou
literaturu.

\vspace{5mm}V Praze dne ....................\hfill
    \begin{tabular}{c}
    ........................................\\
    \autor
    \end{tabular}

% poděkování
\newpage
\thispagestyle{empty}
~
\vfill

{\noindent}{\LARGE Poděkování}

\vspace{5mm}
Děkuji Ing.\ Václavu Šmídlovi, Ph.D. za vedení mé bakalářské práce, prozíravě položené
softwarové otázky a za poukázání na softwarové projekty (Cython, Sphinx), které se ukázaly
být klíčové pro moji bakalářskou práci a softwarový projekt.

\begin{flushright}
\autor
\end{flushright}

% strana s abstraktem
\newpage
\thispagestyle{empty}

\newbox\odstavecbox
\newlength\vyskaodstavce
\newcommand\odstavec[2]{%
    \setbox\odstavecbox=\hbox{%
         \parbox[t]{#1}{#2\vrule width 0pt depth 4pt}}%
    \global\vyskaodstavce=\dp\odstavecbox
    \box\odstavecbox}
\newcommand{\delka}{120mm}

\noindent\begin{tabular}{@{}ll}
  {\em Název práce:} & ~ \\
  \multicolumn{2}{@{}l}{\odstavec{\textwidth}{\bf \nazevcz}} \\[5mm]
  {\em Autor:} & \autor \\[5mm]
  {\em Obor:} & \obor \\
  {\em Druh práce:} & Bakalářská práce \\[5mm]
  {\em Vedoucí práce:} & \odstavec{\delka}{\vedouci \\ \pracovisteVed} \\[5mm]
  {\em Konzultant:} & \odstavec{\delka}{\konzultant \\ \pracovisteKonz} \\[5mm]
  \multicolumn{2}{@{}l}{\odstavec{\textwidth}{{\em Abstrakt:} ~ \abstrCZ \\ }} \\[5mm]
  {\em Klíčová slova:} & \odstavec{\delka}{\klicova} \\[10mm]

  {\em Title:} & ~\\
  \multicolumn{2}{@{}l}{\odstavec{\textwidth}{\bf \nazeven}}\\[5mm]
  {\em Author:} & \autor \\[5mm]
  \multicolumn{2}{@{}l}{\odstavec{\textwidth}{{\em Abstract:} ~ \abstrEN \\ }} \\[5mm]
  {\em Key words:} & \odstavec{\delka}{\keyword}
\end{tabular}
