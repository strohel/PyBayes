\clearpage % so that table of contents mentions correct page
\phantomsection % so that hyperref makes correct reference
\addcontentsline{toc}{chapter}{Introduction}
\pagenumbering{arabic}
\chapter*{Introduction}

Bayesian filtering (or, recursive Bayesian estimation) is a very promising approach to estimation
of dynamic systems; it can be applied to a wide range of real-world problems in areas such as
robotics~\cite{ThrBurFox:05,Gus:02} (tracking, navigation), environmental simulations ---
e.g. tracking of the radioactive plume upon nuclear
accident~\cite{HofSmi:09,HofSmiPech:09,PechHofSmi:09}, econometrics and many more.

While many Bayesian filtering algorithms are simple enough to be implemented in software on
\emph{ad-hoc} basis, it is proposed that a well designed library can bring many advantages such as
ability to combine and interchange individual methods, better performance, programmer convenience
and a tested code-base.

\noindent{}The text is organised as follows:
\begin{enumerate}
	\item Theoretical background of Bayesian filtering is presented in the first chapter along with
	a description of well-known Bayesian filters: the Kalman filter, the particle filter and a
	simple form of the marginalized particle filter.
	\item In chapter 2 a software analysis for a desired library for Bayesian filtering is
	performed: requirements are set up, general approaches to software development are discussed and
	programming languages C++, MATLAB and Python and their implementations are compared. Interesting
	results are achieved when Python is combined with Cython.
	\item The PyBayes library that was developed as a part of this thesis is presented. PyBayes is
	written in Python with an option to use Cython and implements all algorithms presented in the
	first chapter. Performance of PyBayes is measured and confronted with concurrent implementations
	that use different implementation environments.
\end{enumerate}
Bayesian filtering is a subtask of a broader topic of Bayesian decision-making~\cite{Smi:05}; while
decision-making is not covered in this text, we expect the PyBayes library to form a good building
block for implementing decision-making systems.
