\documentclass[a4paper,12pt,oneside]{report}

\newif\ifrelease % new boolean variable release. True = include some fancy content
\releasetrue % and set it

\usepackage[english]{babel}
\usepackage[utf8]{inputenc}
\usepackage[T1]{fontenc}
\usepackage{amsmath}
\usepackage{amssymb}
\usepackage{epsfig}
\usepackage{graphicx}
\ifrelease
	\usepackage{pdfpages}
\fi
\usepackage[pagebackref=true]{hyperref} % tento balicek by mel byt na konci baliku!

\hypersetup{
	pdfauthor={Matej Laitl},
	pdftitle={Implementation environment for Bayesian filtering algorithms}
}

%% Nastavení zrcadla sazby
\usepackage{calc}
\setlength{\textheight}{9in}
\setlength{\textwidth}{6in}
\setlength\oddsidemargin{(\paperwidth-\textwidth)/2 - 1in}
\setlength\evensidemargin{(\paperwidth-\textwidth)/2 - 1in}
\setlength\topmargin{(\paperheight-\textheight-\headheight-\headsep-\footskip)/2 - 1in}

% TODO: je toto potreba?
%\parindent=0pt % odsazení 1. řádku odstavce
%\parskip=7pt   % mezera mezi odstavci

\ifrelease
	\hypersetup{pdfborder={0 0 0}} % no borders around links
\else
	\hypersetup{colorlinks=true} % colour links instead of borders
\fi

\begin{document}

\ifrelease
	\newcommand{\cvut}{České vysoké učení technické v~Praze}
\newcommand{\fjfi}{Fakulta jaderná a fyzikálně inženýrská}
\newcommand{\km}{Katedra matematiky}
\newcommand{\obor}{Inženýrská Informatika}
\newcommand{\zamereni}{Softwarové inženýrství a matematická informatika}

\newcommand{\nazevcz}{Prostředí pro implementaci algoritmů Bayesovské filtrace}
\newcommand{\nazeven}{Implementation environment for Bayesian filtering algorithms}
\newcommand{\autor}{Matěj Laitl}
\newcommand{\rok}{2011}
\newcommand{\vedouci}{Ing. Václav Šmídl, Ph.D.}

\newcommand{\pracovisteVed}{Oddělení adaptivních systémů \\
	Ústav teorie informace a automatizace \\
	Akademie věd České republiky}
\newcommand{\konzultant}{---}
\newcommand{\pracovisteKonz}{}

\newcommand{\klicova}{Bayesovská filtrace, softwarová analýza, Python, Cython} % TODO: dalsi?
\newcommand{\keyword}{Bayesian filtering, software analysis, Python, Cython}
% Abstrakt práce: (cca 7 vět, min. 80 slov)
\newcommand{\abstrCZ}{Bayesovská filtrace je velmi použitelný přístup k odhadování dynamických
systémů, který má mnoho aplikací v robotice, environmentálních simulacích a dalších oborech. Tato
práce předkládá popis Kalmanova filtru, částicového filtru a marginalizovaného částicového filtru,
který využívá myšlenky obou předchozích. Poté je provedena softwarová analýza, která má za úkol
určit nejvhodnější implementační prostředí pro softwarovou knihovnu poskytující metody Bayesovské
filtrace. Nejprve jsou posouzeny obecné přístupy k vývoji software, z nichž je jako nejpoužitelnější
zvolen objektově orientovaný, následuje porovnání programovacích jazyků C++, MATLAB a Python.
Python zvítězí a v je něm napsána knihovna pro Bayesovskou filtraci nazvaná PyBayes; nakonec je
dosaženo výrazného zrychlení Pythonového kódu použitím Cythonu.}
\newcommand{\abstrEN}{Bayesian filtering is a vital approach to dynamic system estimation that has
wide applications in robotics, environmental simulations and much more. This text briefly introduces
the Kalman filter, the particle filter and the marginalized particle filter which combines ideas of
both. Software analysis is performed with the aim to identify the most suitable implementation
environment for a software library for Bayesian filtering. Various paradigms of software development
and are discussed among which the object-oriented programming is chosen; C++, MATLAB and Python
languages are evaluated. Python is determined most appropriate and a Python software library named
PyBayes is developed; dramatic performance gains are measured when Cython is used to speed up
Python.}

%%% zde zacina kresleni dokumentu

% titulní strana
\thispagestyle{empty}

\begin{center}
	{\Large  \bf  \cvut\\[2mm] \fjfi }
	\vspace{10mm}

	\begin{tabular}{c}
	{\bf \km}\\
	{\bf Obor: \obor}\\
	{\bf Zaměření: \zamereni}
	\end{tabular}

	\vspace{10mm} \epsfysize=20mm  \epsffile{cvut-logo-bw-600} \vspace{15mm}

	{\LARGE
	\textbf{\nazevcz}
	\par}

	\vspace{5mm}

	{\LARGE
	\textbf{\nazeven}
	\par}

	\vspace{30mm}
	{\Large BAKALÁŘSKÁ PRÁCE}

\end{center}

\vfill
{\large
\begin{tabular}{rl}
Vypracoval: & \autor\\
Vedoucí práce: & \vedouci\\
Rok: & \rok
\end{tabular}
}

% zadání bakalářské práce
\newpage
\thispagestyle{empty} Před svázáním místo téhle stránky \fbox{vložíte zadání práce} s podpisem
děkana (bude to jediný oboustranný list ve Vaší práci) !!!!

% prohlášení
\newpage
\thispagestyle{empty}
~
\vfill


{\noindent}{\LARGE Prohlášení}

\vspace{0.5cm}
Prohlašuji, že jsem předloženou práci vypracoval samostatně a že jsem uvedl veškerou použitou
literaturu.

\vspace{5mm}V Praze dne ....................\hfill
    \begin{tabular}{c}
    ........................................\\
    \autor
    \end{tabular}

% poděkování
\newpage
\thispagestyle{empty}
~
\vfill

{\noindent}{\LARGE Poděkování}

\vspace{5mm}
Děkuji Ing.\ Václavu Šmídlovi, Ph.D. za vedení mé bakalářské práce, prozíravě položené
softwarové otázky a za poukázání na softwarové projekty (Cython, Sphinx), které se ukázaly
být klíčové pro moji bakalářskou práci a softwarový projekt.

\begin{flushright}
\autor
\end{flushright}

% strana s abstraktem
\newpage
\thispagestyle{empty}

\newbox\odstavecbox
\newlength\vyskaodstavce
\newcommand\odstavec[2]{%
    \setbox\odstavecbox=\hbox{%
         \parbox[t]{#1}{#2\vrule width 0pt depth 4pt}}%
    \global\vyskaodstavce=\dp\odstavecbox
    \box\odstavecbox}
\newcommand{\delka}{120mm}

\noindent\begin{tabular}{@{}ll}
  {\em Název práce:} & ~ \\
  \multicolumn{2}{@{}l}{\odstavec{\textwidth}{\bf \nazevcz}} \\[5mm]
  {\em Autor:} & \autor \\[5mm]
  {\em Obor:} & \obor \\
  {\em Druh práce:} & Bakalářská práce \\[5mm]
  {\em Vedoucí práce:} & \odstavec{\delka}{\vedouci \\ \pracovisteVed} \\[5mm]
  {\em Konzultant:} & \odstavec{\delka}{\konzultant \\ \pracovisteKonz} \\[5mm]
  \multicolumn{2}{@{}l}{\odstavec{\textwidth}{{\em Abstrakt:} ~ \abstrCZ \\ }} \\[5mm]
  {\em Klíčová slova:} & \odstavec{\delka}{\klicova} \\[10mm]

  {\em Title:} & ~\\
  \multicolumn{2}{@{}l}{\odstavec{\textwidth}{\bf \nazeven}}\\[5mm]
  {\em Author:} & \autor \\[5mm]
  \multicolumn{2}{@{}l}{\odstavec{\textwidth}{{\em Abstract:} ~ \abstrEN \\ }} \\[5mm]
  {\em Key words:} & \odstavec{\delka}{\keyword}
\end{tabular}
 % include some fancy start pages
\fi


% obsah
\newpage
\tableofcontents


\chapter*{Introduction} \addcontentsline{toc}{chapter}{Introduction}

TODO motivatin for bayes filtration + a need for a convenient library (rapid prototyping vs. speed)

applications: robotics, navigation, + tracking of toxic plume after radiation accident.

Decision-making being a logical and natural ``next step'' - beyond the scope of this text.

[proposed citations:\cite{ThrBurFox:05,Gus:02,HofSmi:09,HofSmiPech:09,PechHofSmi:09}]


\chapter*{Notation} \addcontentsline{toc}{chapter}{Notation}

Throughout this text, following notation is used

\bigskip

\begin{tabular}{l p{0.8\textwidth}}
	\(\mathbb{N}\) & set of natural numbers \\
	\(\mathbb{R}\) & set of real numbers \\
	\(t\) & discrete time moment; \(t \in \mathbb{N}\) \\
	\(a_t\) & value of quantity \(a\) at time \(t\); \(a_t \in \mathbb{R}^n, n \in \mathbb{N}\) \\
		& unless noted otherwise, \(x_t\) denotes state vector at time \(t\) and \(y_t\) denotes
		  observation vector at time \(t\) \\
	\(a_{i:j}\) & sequence of quantities \((a_i, a_{i+1} \dots a_{j-1}, a_j)\) \\
\end{tabular}


\chapter{Basics of Recursive Bayesian Estimation}

\section{Problem Statement}

Assume a dynamic system described by a hidden state vector of real numbers,
denoted in this text as \(x_t\) ... evolves at discrete time steps according to a known, assumed or
approximated function \(f_t\)
\[ x_t = f_t(x_{t-1}, v_{t-1}) \]
where \(v_t\) is unevitable and random process noise, that comes from ... ... .. .

also assume that we can obseve the system in each time step through an observation \(y_t\) that relates
to the state, but may contain less information and adds further noise. 
\[ y_t = h_t(x_t, w_t) \]
where \(w_t\) is unevitable and random observation noise.

Our goal is to estimate the hidden state \(x_t\) given the observations \(y_{1:t}\) 

resursive = we do not need to keep track of all previous observations/state predictions, only
appropriate \(a_{t-1}\) quantities are needed to estimate \(x_t\).

\section{Theoretical solution}

[Main source for these 2 sections:\cite{AruMasGor:02}]

\section{Kalman Filter}

\section{Particle Filter}

\section{Marginalized Particle Filter}


\chapter{Software analysis}

\section{Requirements}

Ideal library for Bayesian filtering would posses following properties...:

\section{Programming paradigms}

Interpreted vs. Compiled

Object-oriented, procedural and Functional

pass-by reference vs. copy-on-write (Matlab)

\section{Survey of Existing Libraries for Bayesian estimation/decision making}

Brief survey of existing libraries ... not fullfilling all requirements..

The need to implement a new one :-)

\section{C++}

BDM

 - advantages of C (speed, C prevalence (many optimised libraries, BLAS, LAPACK.., OpenMP)

 - disadvantages of C in our ``situation'' (steep learning curve, coplexity because of low-levelness
   high initial barriers (need to have compiler, libraries...), inconveniently long edit/build/test
   process)

\section{Matlab}

BDM (partially?)

 - advantages (popularity, existing toolboxes, rapid development (high-level)

 - disadv: strict copy-on-write, problematic object model (not in original design), difficulties
           interfacing existing C (F) code

\section{Python}

NumPy.... parallelisation (approaches, improvements in Py 3.2) - GIL.. Py3k

\section{Cython}

general info etc... extension types, building, ease of interfacing C (and F) code, .pxd files,
NumPy support

[citations:\cite{BehBraSel:09,Sel:09,BehBraCitDalSelSmi:11}]

\subsection{Gradual Optimisation and Parallelisation}

how can optimisaion be approached (gradually) and why this approach is superior

integrate\_python\_cython example (``100x'' speedup for a special (very simple) case)

\subsection{Parallelisation}

integrate\_python\_cython patched with OpenMP (13x speedup in 16-core system)

prange CEP

\subsection{Pure Python mode}

About it and why it should be used in a hypothetical bayesian python library

\subsection{Limitations}

2 types:

	not-supported code (few cases, but bad, ongoing work)

	not-optimised code (much more work needed, but not hard to fix in most cases)

		- exception handling (functions returning void etc)

		- limitations of pure python mode in regards to traditional .pyx files

\section{Choice}

python/cython was choosen ...


\chapter{The PyBayes Library}

Introduction, general directions, future considerations

+ open development on github, open-source

\section{Interpreted and Compiled}

\section{Library Layout}

[proposed citation: \cite{Smi:05}]

\subsection{Random Variable Meta-representation}

Why it is needed (ref to ProdCPdf)

\subsection{Probability Density Functions}

Nice UML diagrams! (better more smaller UMLs than one big) One for general pdf layut, one for
AbstractGaussPdf family, one for AbstractEmpPdf family

\subsection{Bayesian Filters}

UML

Nice graph of a run of a particle filter (Mirda has the plotting code)

similar of marginalized particle filter? (gausses would be plotted vertically)

[mention this:\cite{Smi:10}]

\section{Documentation, Testing and Profiling}

TODO: move above Library Layout?

Documenting using Sphinx, approach to documentation (mathematician-oriented), math in documentation

Testing - the separation of

- tests: test one class in isolation, quick, determinism (would be good, not achievable)

- stresses: test a great portion of code at once, run longer, non-determinism..

Profiling python/cython - how, existing support in PyBayes

- how to correct profiling-induced overhead

\section{Comparison with BDM}


\chapter*{Conclusion} \addcontentsline{toc}{chapter}{Conclusion}


% použitá literatura
\clearpage % so that the contents link mentions correct page
\phantomsection % so that hyperref makes correct reference
\addcontentsline{toc}{chapter}{\bibname}
\bibliographystyle{plain}
\bibliography{bibliography}

\ifrelease
	% přílohy
	\appendix % aby LaTeX cisloval jinak
	\clearpage % so that the contents link mentions correct page
	\phantomsection % so that hyperref makes correct reference
	\addcontentsline{toc}{chapter}{\appendixname}

	\part*{\appendixname}

	\includepdf[pages=-]{../doc/_build/latex/PyBayes.pdf}
\fi

\end{document}
